\documentclass[a4paper,UKenglish,numberwithinsect,cleveref,thm-restate]{lipics-v2021}
\usepackage[draft,inline,nomargin]{fixme}
\FXRegisterAuthor{lt}{alt}{LT}
\FXRegisterAuthor{hs}{ahs}{Josh}

\newcommand{\Tree}{\mathcal{S}}
\newcommand{\later}{\mathord{\blacktriangleright}}
\newcommand{\laterp}{\mathord{\vartriangleright}}
\newcommand{\PER}{\mathbf{PER}}

\title{Realising Löb's Axiom as Intensional (Guarded) Recursion in Homotopy Type Theory} 
\author{Liang-Ting Chen}%
  {Institute of Information Science, Academia Sinica, Taiwan~\url{https://l-tchen.github.io}}%
  {liang.ting.chen.tw@gmail.com}%
  {https://orcid.org/0000-0002-3250-1331}{}
\author{Hsiang-Shang Ko}%
  {Institute of Information Science, Academia Sinica, Taiwan~\url{https://josh-hs-ko.github.io}}%
  {joshko@iis.sinica.edu.tw}%
  {https://orcid.org/0000-0002-2439-1048}{}
\authorrunning{L.-T.~Chen and H.-S.~Ko}

\Copyright{Liang-Ting Chen and Hsiang-Shang Ko}
\ccsdesc{Theory of computation~Type theory}
\ccsdesc{Theory of computation~Logic}
\ccsdesc{Theory of computation~Constructive mathematics}
\keywords{Löb's theorem, guarded recursion, realizability, homotopy type theory, modal type theory, metaprogramming}

\category{} %optional, e.g. invited paper

\relatedversion{} %optional, e.g. full version hosted on arXiv, HAL, or other respository/website
%\relatedversiondetails[linktext={opt. text shown instead of the URL}, cite=DBLP:books/mk/GrayR93]{Classification (e.g. Full Version, Extended Version, Previous Version}{URL to related version} %linktext and cite are optional

\supplement{The results were formalised in Guarded Cubical Agda with source code available at \url{https://github.com/L-TChen/provability}.}

\funding{This work was supported by the Ministry of Science and Technology of Taiwan under grant MOST~109-2222-E-001-002-MY3.}

\acknowledgements{We want to thank Alex Kavvos and Martin Escard\'o. We are indebted to online platforms such as Zulip teams on Homotopy Type Theory and Agda for generous help and general discussions.}

\nolinenumbers %uncomment to disable line numbering

%\hideLIPIcs  %uncomment to remove references to LIPIcs series (logo, DOI, ...), e.g. when preparing a pre-final version to be uploaded to arXiv or another public repository

%Editor-only macros:: begin (do not touch as author)%%%%%%%%%%%%%%%%%%%%%%%%%%%%%%%%%%
\EventEditors{John Q. Open and Joan R. Access}
\EventNoEds{2}
\EventLongTitle{42nd Conference on Very Important Topics (CVIT 2016)}
\EventShortTitle{CVIT 2016}
\EventAcronym{CVIT}
\EventYear{2016}
\EventDate{December 24--27, 2016}
\EventLocation{Little Whinging, United Kingdom}
\EventLogo{}
\SeriesVolume{42}
\ArticleNo{23}
%%%%%%%%%%%%%%%%%%%%%%%%%%%%%%%%%%%%%%%%%%%%%%%%%%%%%%

\begin{document}

\maketitle

%TODO: Polish abstract
\begin{abstract}
  In this paper we give a positive answer to an open problem recently posed by Kavvos: is there any categorical interpretation of Gödel-Löb axiom~$\Box (\Box A \to A) \to \Box A$ but refuting~$A \to \Box A$?
  We construct an endoexposure of the $\PER$-category of assemblies internal to guarded type theory as the interpretation of the provability modality~$\Box$.
  Of the utmost significance of our understanding for $\Box A$ is that its denotation should involve not only extensions but also intensions, e.g., derivations of a proof, code of programs, or realisers, inspired by its provability reading.
  Our results are established in the framework of ticked cubical type theory---an extension of cubical type theory with guarded recursion and formalised in Guarded Cubical Agda.
  By viewing~$\Box A$ as a type of typed code \`a la Davies and Pfenning, our work is a step towards extending type theories on which proof assistants are based with non-structural but guarded recursion on the meta level in a logically consistent way.
\end{abstract}

\ltnote{Shall we include Chiang as a coauthor?}

\section{Introduction}
\cite{Litak2014}

\paragraph*{Contributions}

\paragraph*{Plan of the paper}

\paragraph*{Related work}
\cite{Visser2019,Beklemishev2006}
\cite{Kavvos2017,Kavvos2020}
\cite{Shamkanov2014,Shamkanov2016a}

\section{Preliminaries}

\subsection{Guarded recursion}

\subsection{Intensionality}
\begin{definition}[Partial equivalence relation]
  
\end{definition}
\begin{definition}[Categories enriched over $\mathbf{PER}$]
  
\end{definition}
\begin{definition}[Exposure]
  also comonadic exposure
\end{definition}


\subsection{Realisability}
\cite{Escardo2017a,Knapp2018}
\begin{definition}[Partial map classifer]
  
\end{definition}
\begin{definition}[Ordered partial combinatory algebra]
\end{definition}
\begin{definition}[Assembly and trackable map]
  
\end{definition}



%\section{Realising provability using topos of trees}
%\subsection{Topos of trees and its internal logic}
%\cite{Birkedal2012,Clouston2016}
\section{Ordered partial combinatory algebra with numbering}

\section{Realising provability using guarded cubical type theory}

\subsection{Guarded cubical type theory}
\begin{enumerate}
  \item Show that there is no $\Box A \to A$ natural in $A$.
  \item Show that there is no $A \to \Box A$ natural in $A$.
  \item Show that $\Box (\Box A \to A) \to \Box A$ natural in $A$ exists.
  \item Show that $A \to \bot$ implies that $A$ must be $\bot$.
\end{enumerate}

\cite{Mogelberg2019a,Veltri2020}

\section{Conclusion}
\paragraph*{Future work}
\begin{enumerate}
  \item topos-theoretic arguments
  \item Mix with \cite{Kavvos2017b}
  \item Consistent version of \cite{Kavvos2017b}
  \item Compare this with \cite{Shamkanov2014} and \cite{Shamkanov2016a}.
\end{enumerate}
\cite{Davies2001b}
%%
%% Bibliography
%%

%% Please use bibtex, 

\bibliographystyle{plainurl}% the mandatory bibstyle
\bibliography{./ref}

\appendix


\section{Styles of lists, enumerations, and descriptions}\label{sec:itemStyles}

List of different predefined enumeration styles:

\begin{itemize}
\item \verb|\begin{itemize}...\end{itemize}|
\item \dots
\item \dots
%\item \dots
\end{itemize}

\begin{enumerate}
\item \verb|\begin{enumerate}...\end{enumerate}|
\item \dots
\item \dots
%\item \dots
\end{enumerate}

\begin{alphaenumerate}
\item \verb|\begin{alphaenumerate}...\end{alphaenumerate}|
\item \dots
\item \dots
%\item \dots
\end{alphaenumerate}

\begin{romanenumerate}
\item \verb|\begin{romanenumerate}...\end{romanenumerate}|
\item \dots
\item \dots
%\item \dots
\end{romanenumerate}

\begin{bracketenumerate}
\item \verb|\begin{bracketenumerate}...\end{bracketenumerate}|
\item \dots
\item \dots
%\item \dots
\end{bracketenumerate}

\begin{description}
\item[Description 1] \verb|\begin{description} \item[Description 1]  ...\end{description}|
\item[Description 2] Fusce eu leo nisi. Cras eget orci neque, eleifend dapibus felis. Duis et leo dui. Nam vulputate, velit et laoreet porttitor, quam arcu facilisis dui, sed malesuada risus massa sit amet neque.
\item[Description 3]  \dots
%\item \dots
\end{description}

\cref{testenv-proposition} and \autoref{testenv-proposition} ...

\section{Theorem-like environments}\label{sec:theorem-environments}

List of different predefined enumeration styles:

\begin{theorem}[Quisque blandit tempus nunc]\label{testenv-theorem}
Fusce eu leo nisi. Cras eget orci neque, eleifend dapibus felis. Duis et leo dui. Nam vulputate, velit et laoreet porttitor, quam arcu facilisis dui, sed malesuada risus massa sit amet neque.
\end{theorem}

\begin{lemma}\label{testenv-lemma}
Fusce eu leo nisi. Cras eget orci neque, eleifend dapibus felis. Duis et leo dui. Nam vulputate, velit et laoreet porttitor, quam arcu facilisis dui, sed malesuada risus massa sit amet neque.
\end{lemma}

\begin{corollary}\label{testenv-corollary}
Fusce eu leo nisi. Cras eget orci neque, eleifend dapibus felis. Duis et leo dui. Nam vulputate, velit et laoreet porttitor, quam arcu facilisis dui, sed malesuada risus massa sit amet neque.
\end{corollary}

\begin{proposition}\label{testenv-proposition}
Fusce eu leo nisi. Cras eget orci neque, eleifend dapibus felis. Duis et leo dui. Nam vulputate, velit et laoreet porttitor, quam arcu facilisis dui, sed malesuada risus massa sit amet neque.
\end{proposition}

\begin{conjecture}\label{testenv-conjecture}
Fusce eu leo nisi. Cras eget orci neque, eleifend dapibus felis. Duis et leo dui. Nam vulputate, velit et laoreet porttitor, quam arcu facilisis dui, sed malesuada risus massa sit amet neque.
\end{conjecture}

\begin{observation}\label{testenv-observation}
Fusce eu leo nisi. Cras eget orci neque, eleifend dapibus felis. Duis et leo dui. Nam vulputate, velit et laoreet porttitor, quam arcu facilisis dui, sed malesuada risus massa sit amet neque.
\end{observation}

\begin{exercise}\label{testenv-exercise}
Fusce eu leo nisi. Cras eget orci neque, eleifend dapibus felis. Duis et leo dui. Nam vulputate, velit et laoreet porttitor, quam arcu facilisis dui, sed malesuada risus massa sit amet neque.
\end{exercise}

\begin{definition}\label{testenv-definition}
Fusce eu leo nisi. Cras eget orci neque, eleifend dapibus felis. Duis et leo dui. Nam vulputate, velit et laoreet porttitor, quam arcu facilisis dui, sed malesuada risus massa sit amet neque.
\end{definition}

\begin{example}\label{testenv-example}
Fusce eu leo nisi. Cras eget orci neque, eleifend dapibus felis. Duis et leo dui. Nam vulputate, velit et laoreet porttitor, quam arcu facilisis dui, sed malesuada risus massa sit amet neque.
\end{example}

\begin{note}\label{testenv-note}
Fusce eu leo nisi. Cras eget orci neque, eleifend dapibus felis. Duis et leo dui. Nam vulputate, velit et laoreet porttitor, quam arcu facilisis dui, sed malesuada risus massa sit amet neque.
\end{note}

\begin{note*}
Fusce eu leo nisi. Cras eget orci neque, eleifend dapibus felis. Duis et leo dui. Nam vulputate, velit et laoreet porttitor, quam arcu facilisis dui, sed malesuada risus massa sit amet neque.
\end{note*}

\begin{remark}\label{testenv-remark}
Fusce eu leo nisi. Cras eget orci neque, eleifend dapibus felis. Duis et leo dui. Nam vulputate, velit et laoreet porttitor, quam arcu facilisis dui, sed malesuada risus massa sit amet neque.
\end{remark}

\begin{remark*}
Fusce eu leo nisi. Cras eget orci neque, eleifend dapibus felis. Duis et leo dui. Nam vulputate, velit et laoreet porttitor, quam arcu facilisis dui, sed malesuada risus massa sit amet neque.
\end{remark*}

\begin{claim}\label{testenv-claim}
Fusce eu leo nisi. Cras eget orci neque, eleifend dapibus felis. Duis et leo dui. Nam vulputate, velit et laoreet porttitor, quam arcu facilisis dui, sed malesuada risus massa sit amet neque.
\end{claim}

\begin{claim*}\label{testenv-claim2}
Fusce eu leo nisi. Cras eget orci neque, eleifend dapibus felis. Duis et leo dui. Nam vulputate, velit et laoreet porttitor, quam arcu facilisis dui, sed malesuada risus massa sit amet neque.
\end{claim*}

\begin{proof}
Fusce eu leo nisi. Cras eget orci neque, eleifend dapibus felis. Duis et leo dui. Nam vulputate, velit et laoreet porttitor, quam arcu facilisis dui, sed malesuada risus massa sit amet neque.
\end{proof}

\begin{claimproof}
Fusce eu leo nisi. Cras eget orci neque, eleifend dapibus felis. Duis et leo dui. Nam vulputate, velit et laoreet porttitor, quam arcu facilisis dui, sed malesuada risus massa sit amet neque.
\end{claimproof}

\end{document}
